
\documentclass[12pt]{article}
\usepackage{amsmath, amssymb, graphicx}

\title{Calcium as a Prerequisite for Grabby Aliens: Earliest Start Times and Observable Horizons}
\author{}
\date{\today}

\begin{document}
\maketitle

\section{Introduction}
We propose that calcium availability is a gating factor for the emergence of fast-evolving, predator-prey-driven biospheres capable of producing technological civilizations.
Without calcium, muscle contraction, rapid nerve impulses, skeletal mineralization, and shell formation are significantly hindered.
On Earth, calcium carbonate and calcium phosphate have enabled both defensive and offensive arms races, accelerating evolution.

\section{Calcium Threshold Epoch}
We define the calcium threshold epoch $t_{\mathrm{Ca}}^\star$ as the first cosmic time when typical star-forming regions achieve a calcium abundance $Z_{\mathrm{Ca}}$ exceeding a critical value $Z_{\mathrm{Ca}}^{\mathrm{crit}}$ sufficient for rapid Ca$^{2+}$-mediated signaling and biomineralization.

\subsection{Earliest Technological Time}
If planets must form after the calcium pulse to trap Ca into biospheres, then the earliest plausible onset of technology is
\begin{equation}
t_{\mathrm{tech}}^{\min} = t_{\mathrm{Ca}}^\star + \tau_{\mathrm{mix}} + \tau_{\mathrm{planet}} + \tau_{\mathrm{bio+intel}},
\end{equation}
where:
\begin{itemize}
\item $\tau_{\mathrm{mix}}$: Delay for ISM mixing and incorporation into disks ($\sim0.2$ Gyr),
\item $\tau_{\mathrm{planet}}$: Time from disk formation to stable rocky planets with oceans ($\sim0.3$ Gyr),
\item $\tau_{\mathrm{bio+intel}}$: Time from planet formation to technological intelligence ($\sim4.5$ Gyr, Earth-like).
\end{itemize}

\subsection{Observable Radii}
Let $t_0$ denote the present cosmic time ($13.8$ Gyr), $v=\beta c$ the expansion speed.
The maximum electromagnetic horizon for earliest civil signals is
\begin{equation}
R_{\mathrm{light}}^{\max} = c\,(t_0 - t_{\mathrm{tech}}^{\min}),
\end{equation}
and the maximum expansion-front radius is
\begin{equation}
R_{\mathrm{exp}}^{\max} = \beta c\,(t_0 - t_{\mathrm{tech}}^{\min}).
\end{equation}

\section{Results}
For baseline parameters ($\tau_{\mathrm{mix}}=0.2$, $\tau_{\mathrm{planet}}=0.3$, $\tau_{\mathrm{bio+intel}}=4.5$), we sweep $t_{\mathrm{Ca}}^\star$:

\begin{center}
\begin{tabular}{|c|c|c|}
\hline
$t_{\mathrm{Ca}}^\star$ (Gyr) & $t_{\mathrm{tech}}^{\min}$ (Gyr) & $R_{\mathrm{light}}^{\max}$ (Gly) \\
\hline
0.8 & 5.8 & 8.0 \\
1.0 & 6.0 & 7.8 \\
1.5 & 6.5 & 7.3 \\
2.0 & 7.0 & 6.8 \\
3.0 & 8.0 & 5.8 \\
\hline
\end{tabular}
\end{center}

The most optimistic case ($t_{\mathrm{Ca}}^\star=0.8$) yields $t_{\mathrm{tech}}^{\min} \approx 5.8$ Gyr and a maximum light sphere of $\approx 8.0$ Gly.

\section{Implications for Grabby Aliens}
In the Grabby Aliens model, this calcium gating shifts the birth-time distribution for civilizations by adding a hard lower bound at $t_{\mathrm{tech}}^{\min}$.
Later calcium availability reduces both the observable light horizon and expansion horizon, potentially decreasing the number of detectable domains.

\end{document}
